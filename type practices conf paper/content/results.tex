% !TeX root = ../main.tex

\section{Study Results}

\subsection{(RQ1) Presence of Dynamic Type-Aware Practices}

\begin{table*}[ht]
	\caption{Number of Dynamic Type-aware Practices in Subject Systems}
	\centering
	\vspace{-10pt}
	\label{tab:instances}
	\begin{tabular}{lcccccc}
		\hline
		\textbf{System} & \textbf{\tabincell{c}{Inconsistent\\Assignment Types}}&	\textbf{\tabincell{c}{Inconsistent\\Argument Types}}& \textbf{\tabincell{c}{Inconsistent\\Variable Types}} & \textbf{\tabincell{c}{Dynamic\\Element Deletion}} & \textbf{\tabincell{c}{Dynamic\\Attribute Deletion}}&\textbf{\tabincell{c}{Dynamic\\Attribute Access }}\\
		\hline
		{\tt borg} & 8 & 3 & 98 & 17 & 0 & 21\\
		{\tt ralph} & 71 & 0 & 13 & 3 & 0 & 88\\
		{\tt fabric} & 6 & 0 & 264 & 15 & 15 & 19\\
		{\tt requests} & 15 & 	0 & 129 & 17 & 0 & 34\\
		{\tt tornado} & 28 & 	7 & 52 & 26 & 5 & 42\\
		{\tt ansible} & 49 & 	6 & 338 & 71 & 1 & 126\\
		{\tt ipython} & 49 & 	9 & 379 & 49 & 6 & 96\\
		{\tt beets} & 45 & 	63 & 201 & 43 & 7 & 32\\
		{\tt scikit-learn} & 75 & 	11 & 983 & 21 & 19 & 101\\
		\hline
		Average & 38 & 11	 & 273	& 29&6 &62\\			
		\hline
	\end{tabular}
\end{table*}

\textbf{Quantitative Research Result.} 
TABLE \ref{tab:instances} presents the number of dynamic type-aware practices detected in the analyzed version of each subject system. Dynamic type-aware practices exist inconsistently in different systems. For example, the most common type of type-aware practices in {\tt ralph} is \emph{Dynamic Attribute Access}, while the most common type in other systems is \emph{Inconsistent Variable Types}. The fact is that one type of dynamic type-aware practices usually gathers together in a system, which is caused by developers' individual coding habits.

Generally, \emph{Inconsistent Variable Types} is the most common type of dynamic type-aware practices, where there are as high as 273 examples in a Python system on average. The fact is that \emph{Inconsistent Variable Types} examples are usually closely linked, because a variable with inconsistent types would propagate to many other variables (e.g., the variables whose values are dependent on it). Therefore, the key of the refactoring of \emph{Inconsistent Variable Types} examples is identifying the root one which propagates the inconsistent types to other variables. Furthermore, we also inspected the inconsistent type sets in detected examples. We discovered that most examples are generated by the combination of \textit{string} with other inconsistent types. It is often the case that the value of a variable is parsed from the user inputs and its type is determined by the sources of the inputs (e.g., XML files, database, command line).

On the contrary, \emph{Inconsistent Argument Types} and \emph{Dynamic Attribute Deletion} rarely occur in Python systems, with only 11 and 6 examples on average. Type annotation is added in Python after the introduction of Python3, which supports annotating the desired types of function arguments. The new feature reduces the misuse of function arguments thus reduces \emph{Inconsistent Argument Types} practices. \emph{Dynamic Attribute Deletion} is seldom performed by developers because they prefer to reserve no longer used attributes according to our observation.


\textbf{Qualitative Research Result.} 
The result of SQ2 in TABLE \ref{tab:surveyresults} shows which types of dynamic type-aware practices are widely used by developers in Python systems. 54\% of Python developers widely perform \emph{Inconsistent Variable Types} practices, because ``duck typing is a widely-used Python idiom'' as they claimed. This finding is consistent with the result of quantitative research. 
Many developers do not perform \emph{Dynamic Attribute Access} widely and they explained that: ``a valid reason of visiting an attribute via reflection is when they write meta code''.
In addition, only 7\% and 3\% of Python developers often use \emph{Dynamic Element Deletion} and \emph{Dynamic Attribute Deletion}, respectively.   


\textbf{Summary.} Generally, \emph{Inconsistent Variable Types} is the most common type of dynamic type-aware practices, while \emph{Dynamic Attribute Deletion} rarely occurs in Python systems.

\subsection{(RQ2) Intents of Dynamic Type-Aware Practices}

\textbf{Quantitative Research Result.}
Grouped by the intents behind the introduction of dynamic type-aware practices, TABLE \ref{tab:introduction} reports the number of commits which introduced dynamic type-aware practice examples into each system according to our mining, inspection, and classification. 
In the table, 60 randomly selected examples were analyzed for each type of dynamic type-aware practices, while only 53 examples of \emph{Dynamic Attribute Deletion} in total were detected in subject systems thus we analyzed all of them. The examples for which the intents are not clear were left out of the analysis.

In total, most of the dynamic type-aware practices were introduced into systems during development phase (accounting for above half of the cases), followed by enhancing an existing feature, introducing a new feature to the system, fixing bugs, performing refactoring and platform update, respectively. Furthermore, we noticed that a commit would introduce multiple examples of dynamic type-aware practices in some cases. For example, a commit that introduced an \emph{Inconsistent Argument Types} example by adding a function call with an inconsistent argument type also introduced \emph{Inconsistent Variable Types} examples due to inconsistent types used in the called function.


When considering the most common intents behind dynamic type-aware practices, i.e., development, enhancement and new feature, this is expected because dynamic type-aware practices are highly useful for rapid development and are also generated for adding complexity followed by feature upgrade. On the contrary, dynamic type-aware practices are seldom used for bug fixing or refactoring. It implies that the studied practices are easy to perform but are not easy to maintain the system. 



\begin{table*}[ht]
	\caption{Number of commits that introduced dynamic type-aware practices into subject systems grouped by different intents}
	\centering
	\vspace{-10pt}
	\label{tab:introduction}
	\begin{tabular}{lcccccccc}
		\hline
		\textbf{Practice} &\textbf{Development} & \textbf{New Feature}&\textbf{Enhancement}&\textbf{Platform Update}&\textbf{Refactoring}&\textbf{Bug Fixing} & \textbf{Total}\\
		\hline
		Inconsistent Assignment Types&47& 4 &4 &0& 2&3&60\\
		Inconsistent Argument Types&44 &3& 9 &1 &1 &2&60\\
		Inconsistent Variable Types&21 &4 &17 &0& 3& 5&60\\
		Dynamic Element Deletion&35& 6 &15 &0 &2& 2&60\\
		Dynamic Attribute Deletion&47& 3 &0 &0& 2 &1 &53\\
		Dynamic Attribute Access&41&10&6 &0& 2 &1&60\\		
		\hline
	\end{tabular}
\end{table*}

\begin{table}[htbp]
	\centering
	\setlength{\tabcolsep}{3.6pt}
	\caption{Ratios of methods which contain dynamic type-aware practices in all system methods and in bug affected methods}
	\vspace{-10pt}
	\label{tab:bug}
	\begin{tabular}{lcccccc}
		\hline
		\multirow{2}{*}{\textbf{System}}&\multicolumn{3}{c}{\textbf{All System Methods}} &\multicolumn{3}{c}{\textbf{Bug Affected Methods}}\\
		\cmidrule(lr){2-4}  \cmidrule(lr){5-7} 
		& \textbf{\textit{\#Total}}& \textbf{\textit{\#Practice}}& \textbf{\textit{Ratio}} & \textbf{\textit{\#Total}}& \textbf{\textit{\#Practice}}& \textbf{\textit{Ratio}} \\
		\hline
		{\tt borg} &879&64	&7.3\%&42&7	&16.7\%\\
		{\tt ralph} &1,066&86&8.1\%&154&31	& 20.1\%\\
		{\tt fabric} &826&121&14.6\%&17&10	&58.9\%	\\
		{\tt requests} &775&87&11.2\%&49&11&22.4\%	\\	
		{\tt  tornado} &3,080&97 &3.1\%&84&6	&7.1\%\\
		{\tt ansible} &2,772&269&9.7\%&1,079&184	&17.1\%\\
		{\tt ipython}&3,012&258&8.6\%&129&17&	13.2\%\\	
		{\tt beets} &3,451&200&5.8\%&358&43 & 12.0\%\\
		{\tt scikit-learn} &5,465&406&7.4\%&1,500&165&	11.0\%\\
		\hline
	\end{tabular}
\end{table}

\textbf{Qualitative Research Result.}
The survey result of SQ3 in TABLE \ref{tab:surveyresults} presents when developers introduce dynamic type-aware practices into Python systems. Generally, Python developers introduce dynamic type-aware practices mainly during development phase (accounting for from 60\% to 78\% of developers), which agrees with the finding from our quantitative research. Developers explained that they ``would like to ensure loose typing and dynamic behaviors'' for rapid development. They admitted that ``some of the ways are not pythonic at all, but reduce many of the downsides of Python compared with static typing''. Developers also would like to perform these practices during software maintenance including adding new feature, feature enhancement, and refactoring. It is slightly different with the quantitative research findings. Developers explained that it often happens when they implement complex functions. The other intents are seldom mentioned by developers, thus are not presented in the table.


\textbf{Summary.} 
Dynamic type-aware practices are introduced by developers into Python systems mainly during early development phase, followed by during software maintenance phase including adding new feature and feature enhancement.




\begin{table*}[ht]
	\centering
	\setlength{\tabcolsep}{3.6pt}
	\caption{Result of Fisher's exact test (\textit{p}-value) and the odds ratio ($OR$)}
	\vspace{-10pt}
	\label{tab:fishertest}
	\begin{tabularwithnotes}{lcccccccccccccc}
		{
			\tnote[]{The number in bold indicates that there is a significant positive relationship between dynamic type-aware practices and bugs ($p$-value$<$0.05 and $OR$$>$1)}
		}
		\hline
		\textbf{System}& \multicolumn{2}{c}{\textbf{\tabincell{c}{Inconsistent\\Assignment Types}}}&	\multicolumn{2}{c}{\textbf{\tabincell{c}{Inconsistent\\Argument Types}}}& \multicolumn{2}{c}{\textbf{\tabincell{c}{Inconsistent\\Variable Types}}} & \multicolumn{2}{c}{\textbf{\tabincell{c}{Dynamic\\Element Deletion}}} & \multicolumn{2}{c}{\textbf{\tabincell{c}{Dynamic\\Attribute Deletion}}}&
		\multicolumn{2}{c}{\textbf{\tabincell{c}{Dynamic\\Attribute Access }}} &\multicolumn{2}{c}{\textbf{Total}} \\
		\cmidrule(lr){2-3}\cmidrule(lr){4-5}\cmidrule(lr){6-7}\cmidrule(lr){8-9}\cmidrule(lr){10-11}\cmidrule(lr){12-13}\cmidrule(lr){14-15}
		&\textbf{\textit{p}-value} & \textbf{\textit{OR}} &\textbf{\textit{p}-value} & \textbf{\textit{OR}}&\textbf{\textit{p}-value} & \textbf{\textit{OR}}&\textbf{\textit{p}-value} & \textbf{\textit{OR}}&\textbf{\textit{p}-value} & \textbf{\textit{OR}}&\textbf{\textit{p}-value} & \textbf{\textit{OR}}&\textbf{\textit{p}-value} & \textbf{\textit{OR}}\\
		\hline
		{\tt borg} &-&-&-&-&-&-&-&-&-&-&-&-&9.81E-02& 2.05	\\
		{\tt ralph} &-&-&-&-&-&-&-&-&-&-&\textbf{8.42E-05}&\textbf{2.95}&\textbf{9.07E-05}&\textbf{2.65}	\\
		{\tt fabric} &-&-&-&-&\textbf{1.98E-06}&\textbf{12.03}&-&-&-&-&-&-&\textbf{5.63E-06}&\textbf{9.97}\\
		{\tt requests} &-&-&-&-&\textbf{1.46E-03}&\textbf{4.17}&-&-&-&-&-&-&\textbf{4.08E-02}&\textbf{2.17}	\\	
		{\tt  tornado} &-&-&-&-&3.82E-01&1.57&3.07E-01&1.88&-&-&6.89E-01&1.17&6.34E-01&1.24\\
		{\tt ansible} &\textbf{2.66E-03}&\textbf{5.06}&-&-&\textbf{3.73E-14}&\textbf{4.39}&\textbf{2.77E-05}&\textbf{3.39}&-&-&\textbf{1.15E-05}&\textbf{2.67}&\textbf{2.50E-21}&\textbf{3.50}\\
		{\tt ipython}&\textbf{3.33E-02}&\textbf{3.71}&-&-&1.76E-01&2.10&1.68E-01&0&-&-&5.80E-01&1.19&1.07E-01&1.50\\	
		{\tt beets} &\textbf{3.79E-03}&\textbf{3.41}&-&-&\textbf{1.67E-04}&\textbf{2.60}&\textbf{3.06E-02}&\textbf{2.31}&-&-&4.14E-01&0.46&\textbf{1.29E-04}&\textbf{2.00}\\
		{\tt scikit-learn} &6.44E-01&1.18&-&-&\textbf{7.48E-03}&\textbf{1.40}&1.61E-02&0.12&-&-&\textbf{2.16E-02}&\textbf{1.76}&\textbf{2.79E-03}&\textbf{1.37}\\
		\hline
		Total & \textbf{7.82E-05}&\textbf{2.04}&5.30E-01& 0.74&\textbf{3.09E-22}& \textbf{2.15}&\textbf{1.18E-02} &\textbf{1.49}&1.65E-01& 0.44&\textbf{4.92E-08}& \textbf{1.87}&\textbf{1.98E-28}&\textbf{1.84}\\
		\hline
	\end{tabularwithnotes}
\end{table*}

\subsection{(RQ3) Relationship with Bugs}

\textbf{Quantitative Research Result.}
From the bug reports of subject systems, we mined the types of exceptions which were reported by users and developers. We found that the most common exceptions resulted from the bugs are \textit{TypeError} and \textit{AttributeError}. They account for 16.4\% and 14.2\% of the bug reports which record the occurred exceptions, respectively. This finding inspires our study of dynamic type-aware practices which may produce \textit{TypeError} and \textit{AttributeError}.

For each subject system, TABLE \ref{tab:bug} reports the ratios of the methods which contain dynamic type-aware practices in all methods and in bug affected methods. For each group of the methods, the table gives the total number of the methods (\emph{\#Total} columns), the number of the methods affected by dynamic type-aware practices (\emph{\#Practice} columns), the ratio of the affected methods (\emph{Ratio} columns, i.e., \emph{\#Practice}/\emph{\#Total}). The result reveals that in the methods which are affected by bugs, the ratios of dynamic type-aware practices are approximately 2 or 3 times higher than the ratios in all methods. We then hypothesize that certain types of dynamic type-aware practices are highly related with bugs in Python systems.

We then examined the hypothesis by applying Fisher's exact test and the odds ratio ($OR$)\cite{b32} to examine the association between dynamic type-aware practices and bugs. The test classifies all methods into four groups: (1) the methods not affected by dynamic type-aware practices or bugs;  (2) the methods only affected by dynamic type-aware practices; (3) the methods only affected by bugs; (4) the methods affected by both dynamic type-aware practices and bugs. Based on the four groups of the methods, Fisher's exact test calculates the $p$-value for evaluating the correlation between the two ways of classification, i.e., dynamic type-aware practices and bugs. The $OR$ indicates the likelihood that an event (e.g., that a method is affected by bugs) occurs, where $OR$$>$1 indicates that bugs are more likely to occur inside the methods affected by dynamic type-aware practices and $OR$=1 indicates an equal probability. In this study, if Fisher's exact test result shows that the bugs occurred in the methods affected by dynamic type-aware practices significantly differ from those in the methods not affected by dynamic type-aware practices ($p$-value$<$0.05), we referred to the $OR$ value to examine whether the presence of dynamic type-aware practices is related with the presence of bugs ($OR$$>$1) or the absence of bugs ($OR$$<$1). 

TABLE \ref{tab:fishertest} reports the result of Fisher's exact test and the odds ratio towards each type of dynamic type-aware practices in each system. The cells of ``\_'' indicate that we think the datasets of dynamic type-aware practices are not sufficient enough to perform Fisher's exact test (i.e., the methods affected by dynamic type-aware practices are less than 20). The cells in bold indicate that there is a significant positive relationship between dynamic type-aware practices and bugs (i.e., $p$-value$<$0.05 and $OR$$>$1). We can see that the positive relationship is discovered in most of subject systems and most types of dynamic type-aware practices. This finding suggests that the methods affected by dynamic type-aware practices are more prone to bugs. The exceptions come from \emph{Inconsistent Argument Types} and \emph{Dynamic Attribute Deletion}, because these dynamic type-aware practices are seldom performed by developers in subject systems.

\textbf{Qualitative Research Result.} 
The last rows of TABLE \ref{tab:surveyresults} report the survey result of SQ4. The astonishing truth is that although dynamic type-aware practices are often introduced into Python systems for particular programming tasks, most of the developers (ranging from 52\% to 67\%) have ever encountered bug issues caused by them. Developers explained that these practices are ``safe in most cases, but may induce maintenance issues'' in complex scenarios during evolution. For example, the \emph{Inconsistent Assignment Types} example in Section \ref{sec:intro} was reported an issue in bug reports. For refactoring, \emph{module\_name} in the first statement was renamed to \emph{api\_to\_module} and the last statement was changed to ``\textit{module\_name=api\_to\_module[api]}'', which makes the code easier to understand. Developers suggested that ``never use the same variable name for two things, even if it's the same type''.

Among all types of dynamic type-aware practices, developers reflected that \emph{Dynamic Attribute Deletion} is more related to bug occurring than other types. This finding is expected because dynamically deleting an attribute affects the operations on all related instances around the whole system. In addition, for the practices of inconsistent types, they said ``it can be a bit error prone when handling different types in the variables'' and they suggested to ``have the type contract explicitly in the documentation''.


\textbf{Summary.}
Most types of dynamic type-aware practices have a significant positive correlation with bugs in Python systems. More than 52\% of developers have realized that dynamic type-aware practices are prone to bugs.